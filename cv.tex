%%%%%%%%%%%%%%%%%%%%%%%%%%%%%%%%%%%%%%%%%%%%%%%%%%%%%%%%%%%%%%%%%%%%%%%%%%%%%%%
% A clean template for an academic CV
%
% Uses tabularx to create two column entries (date and job/edu/citation).
% Defines commands to make adding entries simpler.
%
%%%%%%%%%%%%%%%%%%%%%%%%%%%%%%%%%%%%%%%%%%%%%%%%%%%%%%%%%%%%%%%%%%%%%%%%%%%%%%%

\documentclass[10pt, a4paper]{article}

% Full Unicode support for non-ASCII characters
\usepackage[utf8]{inputenc}

% Useful aliases
\newcommand{\UERJ}{Universidade do Estado do Rio de Janeiro}
\newcommand{\UHM}{University of Hawai`i at M\={a}noa}
\newcommand{\SOEST}{School of Ocean and Earth Science and Technology}
\newcommand{\UHEARTH}{Department of Earth Sciences}
\newcommand{\LIVEARTH}{Department of Earth, Ocean and Ecological Sciences}
\newcommand{\LIVENV}{School of Environmental Sciences}
\newcommand{\LIV}{University of Liverpool}

% Identifying information
\newcommand{\Title}{Curriculum Vit\ae}
\newcommand{\FirstName}{Anmol}
\newcommand{\LastName}{Goel}
\newcommand{\Initials}{A}
\newcommand{\MyName}{\FirstName\ \LastName}
\newcommand{\Me}{\textbf{\LastName, \Initials}}  % For citations
\newcommand{\Email}{anmol.goel@research.iiit.ac.in}
\newcommand{\PersonalWebsite}{www.anmolg.me}
\newcommand{\LabWebsite}{www.precog.iiit.ac.in}
\newcommand{\ORCID}{0000-0001-6123-9515}
\newcommand{\Address}{
  Jane Herdman Building \\ 4 Brownlow Street \\ Liverpool, L69 3GP \\ United Kingdom
}

% Names for citing coauthors



% Template configuration
%%%%%%%%%%%%%%%%%%%%%%%%%%%%%%%%%%%%%%%%%%%%%%%%%%%%%%%%%%%%%%%%%%%%%%%%%%%%%%%

% Disable hyphenation
\usepackage[none]{hyphenat}

% Control the font size
\usepackage{anyfontsize}

% Icon fonts (requires using xelatex or luatex)
\usepackage[fixed]{fontawesome5}
\usepackage{academicons}

% Template variables for styling
\newcommand{\TablePad}{\vspace{-0.4cm}}
\newcommand{\SoftwareTitle}[1]{{\bfseries #1}}
\newcommand{\TableTitle}[1]{{\fontsize{12pt}{0}\selectfont \itshape #1}}

% For fancy and multipage tables
\usepackage{tabularx}
\usepackage{ltablex}

% Define a new environment to place all CV entries in a 2-column table.
% Left column are the dates, right column the entries.
\usepackage{environ}
\NewEnviron{EntriesTable}{
\TablePad
\begin{tabularx}{\textwidth}{@{}p{0.10\textwidth}@{\hspace{0.02\textwidth}}p{0.88\textwidth}@{}}
  \BODY
\end{tabularx}
}
\NewEnviron{EntriesTableExtra}{
\TablePad
\begin{tabularx}{\textwidth}{@{}p{0.10\textwidth}@{\hspace{0.02\textwidth}}p{0.79\textwidth}@{\hspace{0.02\textwidth}}>{\raggedright\arraybackslash}p{0.07\textwidth}}
  \BODY
\end{tabularx}
}

% Macros to add links and mark publications
\newcommand{\DOI}[1]{doi:\href{https://doi.org/#1}{#1}}
\newcommand{\DOILink}[1]{\href{https://doi.org/#1}{doi.org/#1}}
\newcommand{\Website}[1]{\href{https://#1}{#1}}
\newcommand{\Preprint}[1]{\href{https://doi.org/#1}{\faFilePdf}}
\newcommand{\PaperLink}[1]{\href{#1}{\faFilePdf}}
\newcommand{\Youtube}[1]{\href{https://www.youtube.com/watch?v=#1}{\faYoutube}}
\newcommand{\GitHub}[1]{\href{https://github.com/#1}{\faGithub}}
\newcommand{\Data}[1]{\href{#1}{\faChartLine}}
\newcommand{\Slides}[1]{\href{https://#1}{\faTv}}
\newcommand{\SlidesDOI}[1]{\href{https://doi.org/#1}{\faTv}}
\newcommand{\PosterDOI}[1]{\href{https://doi.org/#1}{\faImage}}
\newcommand{\Poster}[1]{\href{#1}{\faImage}}
\newcommand{\OA}{\thinspace\aiOpenAccess\enspace}

% Macros to set the year and duration on the left column
\newcommand{\Duration}[2]{\fontsize{9pt}{0}\selectfont #1 -- #2}
\newcommand{\Year}[1]{\fontsize{9pt}{0}\selectfont #1}
\newcommand{\Ongoing}{on}
\newcommand{\Future}{future}
\newcommand{\Appointment}[4]{\textbf{#1} \newline #2 \newline #3 \newline #4}

% Define command to insert month name and year as date
\usepackage{datetime}
\newdateformat{monthyear}{\monthname[\THEMONTH], \THEYEAR}

% Set the page margins
\usepackage[a4paper,margin=1.5cm,includehead,headsep=5mm]{geometry}

% To get the total page numbers (\pageref{LastPage})
\usepackage{lastpage}

% No indentation
\setlength\parindent{0cm}

% Increase the line spacing
\renewcommand{\baselinestretch}{1.2}
% and the spacing between rows in tables
\renewcommand{\arraystretch}{1.5}

% Remove space between items in itemize and enumerate
\usepackage{enumitem}
\setlist{nosep}

% Use custom colors
\usepackage[usenames,dvipsnames]{xcolor}

% Set fonts (requires compilation with xelatex)
\usepackage{fontspec}
\setmainfont[%
  Path = fonts/notoserif/,
  UprightFont = NotoSerif-Regular,
  BoldFont = NotoSerif-Bold,
  ItalicFont = NotoSerif-Italic,
  Extension = .ttf
]{NotoSerif}



% Set the spacing for sections
\usepackage{titlesec}
\titleformat{\section}
  {\normalfont\Large\mdseries} % format
  {} % label
  {0pt} % separation (left separation for hang)
  {} % text before title
  [\titlerule] % text after title
\titleformat{\subsection}
  {\normalfont\large\mdseries} % format
  {} % label
  {0pt} % separation (left separation for hang)
  {} % text before title

% Disable number of sections. Use this instead of "section*" so that the sections still
% appear as PDF bookmarks. Otherwise, would have to add the table of contents entries
% manually.
\makeatletter
\renewcommand{\@seccntformat}[1]{}
\makeatother

% Set fancy headers
\usepackage{fancyhdr}
\pagestyle{fancy}
\fancyhf{}
\lhead{\fontsize{9pt}{10pt}\selectfont
  \monthyear\today
}
\chead{
  \fontsize{9pt}{10pt}\selectfont
  \MyName
  \hspace{0.2cm} -- \hspace{0.2cm}
  \Title
}
\rhead{\fontsize{9pt}{10pt}\selectfont \thepage{} of \pageref*{LastPage}}
\renewcommand{\headrulewidth}{0pt}

% Metadata for the PDF output and control of hyperlinks
\usepackage[colorlinks=true]{hyperref}
\hypersetup{
  pdftitle={\MyName\ - \Title},
  pdfauthor={\MyName},
  linkcolor=blue,
  citecolor=blue,
  filecolor=black,
  urlcolor=MidnightBlue
}
%%%%%%%%%%%%%%%%%%%%%%%%%%%%%%%%%%%%%%%%%%%%%%%%%%%%%%%%%%%%%%%%%%%%%%%%%%%%%%%


\begin{document}

% No header for the first page
\thispagestyle{empty}

%%%%%%%%%%%%%%%%%%%%%%%%%%%%%%%%%%%%%%%%%%%%%%%%%%%%%%%%%%%%%%%%%%%%%%%%%%%%%%%
\begin{minipage}[t]{0.7\textwidth}
{\fontsize{22pt}{0}\selectfont\MyName}
\end{minipage}
\begin{minipage}[t]{0.3\textwidth}
  \begin{flushright}
    Last updated: \monthyear\today
  \end{flushright}
\end{minipage}
\\[-0.1cm]
\rule{\textwidth}{2pt}
\\[0.1cm]
\begin{minipage}[t]{0.7\textwidth}
    % ORCID: \href{https://orcid.org/\ORCID}{\ORCID}
    % \\
    Email: \href{mailto:\Email}{\Email}
    \\
    Research group: \Website{\LabWebsite}
    \\
    Website: \Website{\PersonalWebsite}
\end{minipage}
\begin{minipage}[t]{0.3\textwidth}
  \begin{flushright}
    Natural Language Processing \\ 
    Computational Social Science \\
    Causal Inference
  \end{flushright}
\end{minipage}



%%%%%%%%%%%%%%%%%%%%%%%%%%%%%%%%%%%%%%%%%%%%%%%%%%%%%%%%%%%%%%%%%%%%%%%%%%%%%%%
\section{Education}

\begin{EntriesTable}
  \Duration{2022}{\Ongoing}  &
  \textbf{MS by Research (CSE)}, IIIT Hyderabad | CGPA: 9.5/10
  \\
  \Duration{2017}{2021}  &
  \textbf{Bachelor of Technology (CSE)}, Guru Gobind Singh Indraprashta University, Delhi | CGPA: 8.85/10 
\end{EntriesTable}


%%%%%%%%%%%%%%%%%%%%%%%%%%%%%%%%%%%%%%%%%%%%%%%%%%%%%%%%%%%%%%%%%%%%%%%%%%%%%%%
\section{Experience}

\begin{EntriesTable}
  \Duration{Jun'21}{\Ongoing}  &
  \Appointment{Research Associate - IIIT Hyderabad}{Centre for Computational Social Science}{Advised by Prof. Ponnurangam Kumaraguru}{Worked on Long Document Summarization and linguistically-motivated NLP algorithms for Indian Languages.}
  \\
  \Duration{Jan'20}{May'20}  &
  \Appointment{Research Intern - IIIT Delhi}{Precog Lab}{Advised by Prof. Ponnurangam Kumaraguru}{Worked on canonicalization of industry-scale knowledge graphs and studied linguistic evolution through lexical semantic change.}
  \\
  \Duration{May'19}{Aug'19}  &
  \Appointment{Research Intern}{Wellowise}{Advised by Dr. Saher Mehdi}{Developed large scale medical domain NLP and graph based algorithms for disease diagnosis.}
  \\
  \Duration{May'18}{Jun'18}  &
  \Appointment{Summer Intern}{Defence Research and Development Organisation (DRDO)}{Advised by Dr. Chanchal Sharma}{Explored methods to model real-time epitaxy data from Solid State Physics Laboratory.}
\end{EntriesTable}






%%%%%%%%%%%%%%%%%%%%%%%%%%%%%%%%%%%%%%%%%%%%%%%%%%%%%%%%%%%%%%%%%%%%%%%%%%%%%%%
\section{Publications}

\subsection{Conference Proceedings}

\begin{EntriesTableExtra}
\Year{2022}  &
  \textbf{Anmol Goel}, Charu Sharma, Ponnurangam Kumaraguru.
  An Unsupervised, Geometric and Syntax-aware Quantification of Polysemy.
  \emph{In Empirical Methods in Natural Language Processing (EMNLP)}.
  &
  \PaperLink{https://precog.iiit.ac.in/pubs/emnlp_polysemy.pdf}
  \\
  ~ &
  Prashant Kodali, \textbf{Anmol Goel}, Monojit Choudhury, Manish Shrivastava, Ponnurangam Kumaraguru.
  SyMCoM - Syntactic Measure of Code Mixing A Study Of English-Hindi Code-Mixing.
  \emph{In Findings of Association for Computational Linguistics (ACL)}.
  \DOI{10.18653/v1/2022.findings-acl.40}
  &
  \OA
  \PaperLink{https://aclanthology.org/2022.findings-acl.40/}
  \Slides{https://precog.iiit.ac.in/pubs/SyMCoM-ACL2022-presentation.pdf}
  \Poster{https://precog.iiit.ac.in/pubs/SYMCOM-POSTER.pdf}
  \\
  ~ &
  Arnav Kapoor, Mudit Dhawan, \textbf{Anmol Goel}, Arjun TH, Akshala Bhatnagar, Vibhu Agrawal, Amul Agrawal, Arnab Bhattacharya, Ponnurangam Kumaraguru, Ashutosh Modi.
  HLDC: Hindi Legal Documents Corpus.
  \emph{In Findings of Association for Computational Linguistics (ACL)}.
  \DOI{10.18653/v1/2022.findings-acl.278}.
  &
  \OA
  \GitHub{Exploration-Lab/HLDC}
  \PaperLink{https://aclanthology.org/2022.findings-acl.278/}
  \\
  ~ &
  Udit Arora, Nidhi Goyal, \textbf{Anmol Goel}, Niharika Sachdeva, Ponnurangam Kumaraguru.
  Ask It Right! Identifying Low-Quality questions on Community Question Answering Services.
  \emph{In International Joint Conference on Neural Networks (IJCNN)}.
  \DOI{10.1109/IJCNN55064.2022.9892454}.
  &
  \PaperLink{https://ieeexplore.ieee.org/document/9892454}
  \Data{https://precog.iiit.ac.in/resources.html}
  \\
\Year{2021}  &
  Nidhi Goyal, Niharika Sachdeva, \textbf{Anmol Goel}, Jushaan Kalra, Ponnurangam Kumaraguru. 
  KCNet: Kernel-based Canonicalization Network for entities in Recruitment Domain.
  \emph{In 30th International Conference on Artificial Neural Networks (ICANN)}.
  \DOI{10.1007/978-3-030-86340-1\_13}.
  &
  \PaperLink{https://precog.iiit.ac.in/pubs/2021_July_KCNet.pdf}
  \Slides{https://precog.iiit.ac.in/pubs/2021_July_KCNet-slides.pdf}
\end{EntriesTableExtra}


\subsection{Workshop Publications}

\begin{EntriesTableExtra}
\Year{2021}  &
  \textbf{Anmol Goel}, Ponnurangam Kumaraguru.
  Detecting Lexical Semantic Change across Corpora with Smooth Manifolds (Student Abstract).
  \emph{In AAAI 2021 Student Track. Finalist for Best Student Paper Award.}
  \DOI{10.1609/aaai.v35i18.17888}
  &
  \OA
  \PaperLink{https://ojs.aaai.org/index.php/AAAI/article/view/17888}
  \Slides{https://precog.iiit.ac.in/pubs/2020_Nov_smooth_manifolds_for_lexical_semantic_change.pdf}
  \\
  ~ &
  \textbf{Anmol Goel}, Ponnurangam Kumaraguru.
  A Geometric Measure of Polysemy in Hindi Language.
  \emph{In ACM CODS COMAD}
  \DOI{10.1145/3430984.3431044}
  &
  \OA
  \PaperLink{https://dl.acm.org/doi/abs/10.1145/3430984.3431044}
  \Slides{https://precog.iiit.ac.in/pubs/2021_Jan_Geometric_Polysemy_Slides.pdf}
  \\
  ~ &
  Devansh Gautam, Prashant Kodali, Kshitij Gupta, \textbf{Anmol Goel}, Manish Shrivastava, Ponnurangam Kumaraguru.
  CoMeT: Towards Code-Mixed Translation Using Parallel Monolingual Sentences.
  \emph{In CALCS colocated with NAACL}
  \DOI{10.18653/v1/2021.calcs-1.7}
  &
  \OA
  \PaperLink{https://aclanthology.org/2021.calcs-1.7/}

\end{EntriesTableExtra}


%%%%%%%%%%%%%%%%%%%%%%%%%%%%%%%%%%%%%%%%%%%%%%%%%%%%%%%%%%%%%%%%%%%%%%%%%%%%%%%
\section{Academic Service}

\subsection{Reviewer}

\begin{itemize}
  \item EMNLP 2022
  \item EMNLP 2022 Industry Track
  \item NAACL 2022 Industry Track
\end{itemize}

\subsection{Sub Reviewer}

\begin{itemize}
  \item ICML 2022
  \item ACML 2022
  \item CODS COMAD 2022
\end{itemize}


%%%%%%%%%%%%%%%%%%%%%%%%%%%%%%%%%%%%%%%%%%%%%%%%%%%%%%%%%%%%%%%%%%%%%%%%%%%%%%%
\section{Teaching}

\subsection{Teaching Assistant}

\begin{EntriesTableExtra}
  \Year{2022}  &
  Summer Institute in Computational Social Science (SICSS)
  \newline
  Handled tutorials, hands-on sessions and guest lectures
  \newline
  \textit{IIIT Hyderabad}
  & ~
\end{EntriesTableExtra}

%%%%%%%%%%%%%%%%%%%%%%%%%%%%%%%%%%%%%%%%%%%%%%%%%%%%%%%%%%%%%%%%%%%%%%%%%%%%%%%
\section{Awards \& Honors}

\begin{EntriesTable}
  \Year{2022} &
  EMNLP Travel Grant worth \$1500 USD
  \\
  \Year{2020}  &
  NeurIPS, ICML and ACL Travel Grant
  \\
  \Year{2020}  &
  Accepted into RegML Summer School
  \\
  \Year{2021}  &
  Received Udacity Intel AI Scholarship
\end{EntriesTable}



%%%%%%%%%%%%%%%%%%%%%%%%%%%%%%%%%%%%%%%%%%%%%%%%%%%%%%%%%%%%%%%%%%%%%%%%%%%%%%%
\section{Miscellaneous}


\subsection{Languages}

\TablePad
\begin{tabularx}{\textwidth}{@{}p{0.15\textwidth} p{0.85\textwidth}@{}}
  Hindi & Native
  \\
  English & CBSE 12th Standard
  \\
  French & CBSE 10th Standard
  % IELTS: CEFR Level C2 (mastery or proficiency) obtained in 2019
\end{tabularx}

%%%%%%%%%%%%%%%%%%%%%%%%%%%%%%%%%%%%%%%%%%%%%%%%%%%%%%%%%%%%%%%%%%%%%%%%%%%%%%%
\section{References}
\begin{tabularx}{\textwidth}{@{}p{0.35\textwidth}p{0.65\textwidth}@{}}
  Dr. Ponnurangam Kumaraguru & Professor at IIIT Hyderabad
  \\
  Dr. Monojit Choudhury & Principal Applied Scientist at Microsoft (Turing) India
  \\
  Dr. Saptarshi Ghosh & Assistant Professor at IIT Kharagpur
  \\
  Dr. Charu Sharma & Assistant Professor at IIIT Hyderabad 
\end{tabularx}

%%%%%%%%%%%%%%%%%%%%%%%%%%%%%%%%%%%%%%%%%%%%%%%%%%%%%%%%%%%%%%%%%%%%%%%%%%%%%%%
\section{Glossary}

These are the meanings of the symbols used throughout this document:
\\
\TablePad
\begin{tabularx}{\textwidth}{@{}p{0.03\textwidth} p{0.97\textwidth}@{}}
  \aiOpenAccess & Indicates that a publication is open-access
  \\
  \faGithub & Link to a code repository on GitHub
  \\
  \faFilePdf & Link to an open-access PDF, usually a preprint or postprint
  \\
  \faYoutube & Link to a video on YouTube
  \\
  \faChartLine & Link to a data archive
  \\
  \faTv & Link to presentation slides
  \\
  \faImage & Link to a poster
\end{tabularx}

\end{document}
